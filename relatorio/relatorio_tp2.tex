% Arquivo principal do relatório TP2 - Busca Competitiva
% Template baseado no formato AAAI

\documentclass[letterpaper]{article}
\usepackage{aaai}
\usepackage{times}
\usepackage{helvet}
\usepackage{courier}
\usepackage[utf8]{inputenc}
% \usepackage[portuguese]{babel}  % Comentado: precisa instalar pacote babel-portuguese
\usepackage{graphicx}
\usepackage{booktabs}
\usepackage{multirow}
\usepackage{url}

\frenchspacing
\setlength{\pdfpagewidth}{8.5in}
\setlength{\pdfpageheight}{11in}

\pdfinfo{
/Title (TP2: Busca Competitiva - Ligue-4)
/Author (Seu Nome)
}

\setcounter{secnumdepth}{0}

\begin{document}

\title{TP2: Busca Competitiva em Ligue-4\\Utilizando Minimax, Poda Alfa-Beta e Iterative Deepening}

\author{Seu Nome\\
Universidade Federal de Minas Gerais\\
Departamento de Ciência da Computação\\
\texttt{seu.email@ufmg.br}
}

\maketitle

\begin{abstract}
\begin{quote}
Este trabalho apresenta a implementação de um agente de inteligência artificial para jogar Ligue-4 (Connect Four) utilizando algoritmos de busca adversarial. Foram implementados os algoritmos Minimax com profundidade limitada, poda Alfa-Beta e Iterative Deepening, combinados com uma função heurística que avalia posições do tabuleiro. Os experimentos demonstram a eficácia da poda Alfa-Beta em reduzir o número de nós explorados e a capacidade do Iterative Deepening em aproveitar melhor o tempo disponível. Os resultados mostram que o agente implementado é capaz de vencer consistentemente contra um jogador aleatório e apresenta desempenho competitivo em diferentes configurações de profundidade e tempo.
\end{quote}
\end{abstract}

\section{Introdução e Objetivo}

O objetivo deste trabalho é implementar e avaliar algoritmos de busca adversarial para o jogo Ligue-4 (Connect Four). Especificamente, implementamos:

\begin{itemize}
\item Algoritmo Minimax com profundidade limitada e função heurística
\item Poda Alfa-Beta para otimização da busca
\item Iterative Deepening com limite de tempo
\item Função heurística de avaliação de posições
\end{itemize}

O trabalho visa demonstrar a eficácia desses algoritmos e analisar trade-offs entre profundidade de busca, tempo de execução e qualidade das decisões.

\section{Metodologia}

\subsection{Evolução do Agente}

O agente foi desenvolvido incrementalmente, seguindo as seguintes etapas:

\subsubsection{Baseline: Agente Aleatório}
Inicialmente, validamos o ambiente com um agente que escolhe jogadas aleatoriamente entre as colunas válidas.

\subsubsection{Minimax com Profundidade Limitada}
Implementamos o algoritmo Minimax clássico com profundidade máxima configurável. Para estados terminais, atribuímos valores extremos (vitória = $+\infty$, derrota = $-\infty$, empate = 0). Para estados não terminais, utilizamos uma função heurística.

\subsubsection{Poda Alfa-Beta}
Adicionamos a poda Alfa-Beta ao Minimax, mantendo a mesma decisão ótima mas reduzindo significativamente o número de nós explorados.

\subsubsection{Iterative Deepening}
Implementamos Iterative Deepening que explora profundidades progressivamente (1, 2, 3, ...) até atingir a profundidade máxima ou estourar o limite de tempo, mantendo sempre a melhor jogada conhecida.

\subsection{Função Heurística}

Nossa função heurística \texttt{evaluate()} combina três componentes:

\begin{enumerate}
\item \textbf{Valorização do centro}: Peças nas colunas centrais (2, 3, 4) recebem pontos adicionais, com a coluna 3 (central) recebendo maior peso.
\item \textbf{Contagem de sequências}: Sequências de 2 peças recebem peso 1, sequências de 3 peças recebem peso 10.
\item \textbf{Detecção de ameaças}: Sequências de 3 peças que podem virar 4 (com espaço vazio acessível) recebem peso muito alto (500 para ameaças próprias, -1000 para ameaças do oponente).
\end{enumerate}

\subsection{Decisões de Projeto}

\begin{itemize}
\item \textbf{Ordenação de jogadas}: Implementamos ordenação que prioriza colunas centrais, melhorando a eficiência da poda Alfa-Beta.
\item \textbf{Verificação de acessibilidade}: A detecção de ameaças considera apenas espaços vazios acessíveis (que podem receber peças por gravidade).
\item \textbf{Contadores de estatísticas}: Implementamos contadores para número de nós visitados e nós podados, essenciais para análise experimental.
\end{itemize}

\section{Experimentos e Resultados}

\subsection{Minimax vs Aleatório}

Realizamos experimentos comparando Minimax com diferentes profundidades (2, 3, 4, 5) contra um jogador aleatório.

% TABELA 1: Minimax vs Aleatório
\begin{table*}[h]
\centering
\caption{Resultados: Minimax vs Aleatório}
\begin{tabular}{lcccc}
\toprule
Profundidade & Taxa Vitória & Tempo Médio (ms) & Estados Visitados \\
\midrule
2 & XX\% & XX & XX \\
3 & XX\% & XX & XX \\
4 & XX\% & XX & XX \\
5 & XX\% & XX & XX \\
\bottomrule
\end{tabular}
\end{table*}

\subsection{Alfa-Beta vs Minimax (sem poda)}

Comparamos a versão com poda Alfa-Beta contra a versão sem poda, mantendo a mesma profundidade.

% TABELA 2: Alfa-Beta vs Minimax
\begin{table*}[h]
\centering
\caption{Resultados: Alfa-Beta vs Minimax (sem poda)}
\begin{tabular}{lcccc}
\toprule
Profundidade & Taxa Vitória & Tempo Médio (ms) & Estados Visitados \\
\midrule
2 & XX\% & XX & XX (AB) / XX (Minimax) \\
3 & XX\% & XX & XX (AB) / XX (Minimax) \\
4 & XX\% & XX & XX (AB) / XX (Minimax) \\
5 & XX\% & XX & XX (AB) / XX (Minimax) \\
\bottomrule
\end{tabular}
\end{table*}

\subsection{Iterative Deepening vs Alfa-Beta}

Avaliamos o Iterative Deepening com limites de tempo de 1s e 2s contra Alfa-Beta com profundidade fixa.

% TABELA 3: Iterative Deepening vs Alfa-Beta
\begin{table*}[h]
\centering
\caption{Resultados: Iterative Deepening vs Alfa-Beta}
\begin{tabular}{lcccc}
\toprule
Limite Tempo & Taxa Vitória & Tempo Médio (ms) & Prof. Média & Estados Visitados \\
\midrule
1s & XX\% & XX & XX & XX \\
2s & XX\% & XX & XX & XX \\
\bottomrule
\end{tabular}
\end{table*}

\subsection{IA vs Jogador Humano}

Joguei X partidas contra a IA implementada. Percepções qualitativas:

\begin{itemize}
\item \textbf{Forças}: A IA demonstra boa capacidade de bloquear ameaças imediatas e criar oportunidades de vitória.
\item \textbf{Fraquezas}: [A completar após mais partidas]
\end{itemize}

\section{Discussão}

\subsection{Análise dos Resultados}

[Análise crítica dos resultados obtidos, incluindo:
\begin{itemize}
\item Impacto da profundidade na qualidade das decisões
\item Eficiência da poda Alfa-Beta
\item Vantagens do Iterative Deepening
\item Trade-offs entre tempo e qualidade
\end{itemize}]

\subsection{Limitações}

\begin{itemize}
\item A função heurística pode ser melhorada com mais componentes
\item Profundidades maiores (5+) são computacionalmente caras
\item Não implementamos tabela de transposições (otimização futura)
\end{itemize}

\section{Conclusão}

Este trabalho demonstrou a implementação bem-sucedida de algoritmos de busca adversarial para Ligue-4. A poda Alfa-Beta mostrou-se essencial para viabilizar buscas em profundidades maiores, enquanto o Iterative Deepening permitiu melhor aproveitamento do tempo disponível. A função heurística implementada, embora simples, mostrou-se eficaz para guiar a busca.

\textbf{Ideias de melhorias futuras}:
\begin{itemize}
\item Implementar tabela de transposições
\item Melhorar função heurística com mais componentes
\item Otimizações adicionais para competição
\end{itemize}

% Bibliografia (se necessário)
% \begin{thebibliography}{}
% \bibitem{ref1} Referência 1
% \end{thebibliography}

\end{document}

